%---------- Inleiding ---------------------------------------------------------

\section{Introductie}%
\label{sec:introductie}

Op vlak van mobiele app-ontwikkeling is er altijd een tweestrijd tussen het ontwikkelen voor Android en/of iOS. Het ontwikkelen voor beide platformen op een Native-aanpak voor dezelfde applicatie is echter een kostelijke en tijdrovende onderneming. De meeste Android applicaties worden geschreven met een combinatie van Java en Kotlin, terwijl iOS applicaties geschreven zijn in de door Apple ontwikkelde programmeertaal Swift. Het Native programmeren, d.w.z. het ontwikkelen van een applicatie specifiek bedoeld om te draaien op een bepaald platform, zorgt ervoor dat Swift-applicaties niet kunnen draaien op Android-platformen en omgekeerd. Vandaar het kostelijke aspect van het ontwikkelen voor beide platformen.

Cross-platform ontwikkelen biedt een oplossing voor dit probleem. De applicatie wordt hierbij één keer geschreven en kan vervolgens op verschillende platformen draaien. Dit wordt dan geprogrammeerd in talen die op verschillende platformen kunnen draaien zoals bijvoorbeeld JavaScript en C#. Hierbij is het grote nadeel dat de applicatie niet volledig is geoptimaliseerd voor het platform waarop het draait. Dit kan zelf leiden tot een tragere applicatie en een slechtere gebruikerservaring.

Dit bachelorproef zal zich focussen op het vergelijken van twee cross-platform frameworks, namelijk React Native en Ionic, op het Android-platform. React Native is een framework dat ontwikkeld is door Facebook en Ionic een framework dat ontwikkeld is door Drifty Co. Bij deze vergelijking zal er specifiek gekeken worden naar de performantie van de applicaties op basis van CPU-verbruik, laadtijden en geheugengebruik. De keuze voor deze specifieke frameworks is gebaseerd op de populariteit van deze frameworks, aangezien deze twee van de meest gebruikte cross-platform frameworks zijn.  Beide frameworks zijn bovendien open-source en gratis te gebruiken, wat de toegankelijkheid juist vergroot.

Hiervoor zal er een Proof-of-Concept worden uitgevoerd waarbij er een identieke applicatie zal worden ontwikkeld in React Native en Ionic. Deze applicaties zullen vervolgens getest worden op eerder vermelde performantie-aspecten. De resultaten van deze testen zullen tot slot worden geanalyseerd en vergeleken. Er zullen verschillende versies van deze applicaties worden ontwikkeld. Dit om uitvoerig en diepgaand te testen welke aspecten in welk framework een hogere impact hebben op de performantie. Zo kan er een uitgebreid beeld worden geschetst wat de voor- en nadelen zijn van beide frameworks.

Het onderzoek is gericht op ontwikkelaars en kleinere of middelgrote organisaties die met een beperkt budget hun applicatie op een zo groot mogelijke markt willen uitbrengen. Het onderzoek zal een antwoord bieden op de vraag welk framework het meest geschikt is voor het ontwikkelen van een performante applicatie op het Android-platform. Dit zal een meerwaarde bieden voor deze doelgroep, aangezien zij op basis van dit onderzoek een weloverwogen keuze kunnen maken voor het ontwikkelen van hun applicatie.


%---------- Stand van zaken ---------------------------------------------------

\section{State-of-the-art}%
\label{sec:state-of-the-art}

\paragraph{Native vs Cross-Platform}
\newline
Allereerst is het belangrijk om het verschil tussen native en cross-platform applicaties te omkaderen. Zoals eerder vermeld, verwijst Native programmeren naar het ontwikkelen van een applicatie bedoeld om te draaien op een gekozen platform. Dit betekent dat de applicatie geschreven is in de programmeertaal die specifiek bedoeld is voor het platform waarop het draait. Denk maar bijvoorbeeld aan Java en Kotlin voor Android. Cross-platformen daarentegen maken gebruik van onafhankelijke frameworks en talen die op verschillende platformen kunnen draaien (bv JavaScript). Deze worden vervolgens omgezet naar Native code en Native componenten specifiek aan het platform waarop het draait (BRON2).

Een voorbeeld van een Cross-Platform applicatie is een Hybride applicatie. Hierbij wordt er gesteund op webtechnologieën zoals HTML, CSS en JavaScript. Deze applicatie draait vervolgens in een native container van het specifieke platform. Dit wil zeggen dat de applicatie in de browser-engine van het desbetreffende platform draait (bron6). Bij Android is dit het WebView component, bij iOS wordt het UIWebView component gebruikt (BRON4). Aan de hand van een abstractielaag wordt er vervolgens toegang verkregen tot de mogelijkheden van het apparaat via JavaScript-API's (Bron6). Deze abstractielaag kan echter wel voor tragere prestaties zorgen, vooral op de CPU, aangezien de applicatie niet volledig "Native" is en als gevolg niet volledig geoptimaliseerd is aan het platform++++(Bron1).

\paragraph{React Native vs Ionic}
\newline
Voor dit onderzoek is er specifiek gekozen voor React Native en Ionic. Beide zijn cross-platform frameworks die gebruik maken van JavaScript als basis. Ionic is een open-source Hybrid framework ontwikkeld door Drifty Co? en laat toe om te werken met populaire JavaScript frameworks zoals Angular, React en Vue(https://ionicframework.com/docs#license).+++++ (2019 rework) (++++++ hoe weergeven??????)+++++++++++++++++++
React Native is, net zoals Ionic, een open-source framework. Het is ontwikkeld door Facebook en laat toe om te werken met React(bron3). Uit het brononderzoek bleek al snel dat er een paar misconcepties zijn over React Native. Zo wordt het vaak verkeerd beschouwd als een volledig "Native" framework. Het is en blijft Hybrid door het gebruik van JavaScript en JSX (een extensie van JavaScript) componenten. Het "Native" aspect in de naam verwijst naar het feit dat het framework gebruik maakt van Native UI-elementen (bron2). De JSX-componenten worden als het ware omgezet naar Native UI-elementen. Vandaar dat dit soms verkeerd wordt gecategoriseerd.
VM???????????????????????
SAMENVATTENDE ZIN

\paragraph{Architectuur van Android}
\newline
https://developer.android.com/guide/platform

IMAGE
++++++++++++
Het Android platform is een mobiele architectuur ontwikkeld door Google. Deze wordt opgedeeld in de volgende lagen:
  - System Apps: Dit zijn een verzameling standaardapplicaties die worden meegeleverd met het Android-platform. ++++ email sms etc 
  - Java API Framework: Deze laag biedt een set van functionaliteiten die gebruikt worden door +++++
  - Native C/C++ Libraries: Hierin bevinden zich de kerncomponenten en services van Android geschreven in C en C++ code. Daarnaast bevat dit ook de Java OpenGL API die instaat voor het renderen van 2D en 3D graphics.
  - Android Runtime (ART): Dit is de virtuele machine van Android doe de applicaties uitvoert. Enkele features van ART zijn onder andere garbage collection, just-in-time (JIT) compilatie en ahead-of-time (AOT) compilatie.
  - Hardware Abstraction Layer (HAL): Deze laag biedt een interface naar de onderliggende hardware. Deze zorgt ervoor dat de bovenliggende lagen niet rechtstreeks moeten communiceren met de hardware. Het bestaat uit verschillende library modules die elk een specifiek hardwarecomponent vertegenwoordigen.
  - Linux Kernel: Ook wel de kern van het Android-platform. Deze biedt een hardware-abstractielaag, een beveiligingsmodel, process management, geheugen management en een netwerkstack. Daarnaast biedt het ook drivers voor de verschillende hardwarecomponenten+++++.



%Dit state-of-the-art overzicht biedt een basis voor verder onderzoek naar de prestatieoptimalisatie van mobiele apps met specifieke aandacht voor React Native en Ionic op het Android-platform.


%Hier beschrijf je de \emph{state-of-the-art} rondom je gekozen onderzoeksdomein, d.w.z.\ een inleidende, doorlopende tekst over het onderzoeksdomein van je bachelorproef. Je steunt daarbij heel sterk op de professionele \emph{vakliteratuur}, en niet zozeer op populariserende teksten voor een breed publiek. Wat is de huidige stand van zaken in dit domein, en wat zijn nog eventuele open vragen (die misschien de aanleiding waren tot je onderzoeksvraag!)?

%Je mag de titel van deze sectie ook aanpassen (literatuurstudie, stand van zaken, enz.). Zijn er al gelijkaardige onderzoeken gevoerd? Wat concluderen ze? Wat is het verschil met jouw onderzoek?

%Verwijs bij elke introductie van een term of bewering over het domein naar de vakliteratuur, bijvoorbeeld~\autocite{Hykes2013}! Denk zeker goed na welke werken je refereert en waarom.

%Draag zorg voor correcte literatuurverwijzingen! Een bronvermelding hoort thuis \emph{binnen} de zin waar je je op die bron baseert, dus niet er buiten! Maak meteen een verwijzing als je gebruik maakt van een bron. Doe dit dus \emph{niet} aan het einde van een lange paragraaf. Baseer nooit teveel aansluitende tekst op eenzelfde bron.

%Als je informatie over bronnen verzamelt in JabRef, zorg er dan voor dat alle nodige info aanwezig is om de bron terug te vinden (zoals uitvoerig besproken in de lessen Research Methods).

% Voor literatuurverwijzingen zijn er twee belangrijke commando's:
% \autocite{KEY} => (Auteur, jaartal) Gebruik dit als de naam van de auteur
%   geen onderdeel is van de zin.
% \textcite{KEY} => Auteur (jaartal)  Gebruik dit als de auteursnaam wel een
%   functie heeft in de zin (bv. ``Uit onderzoek door Doll & Hill (1954) bleek
%   ...'')

Je mag deze sectie nog verder onderverdelen in subsecties als dit de structuur van de tekst kan verduidelijken.

%---------- Methodologie ------------------------------------------------------
\section{Methodologie}%
\label{sec:methodologie}

Deze bachelorproef begint met een literatuurstudie van Cross-Platform ontwikkeling en de frameworks React Native en Ionic. Hierbij wordt er gekeken naar de architectuur van beide en met elkaar vergeleken om zo een enkele verklaringen te geven voor mogelijke performantieverschillen. Daarnaast wordt de Android architectuur dieper bekeken om dit te linken aan ++++. Deze fase zal een drietal weken duren.

In een volgende fase wordt er, op basis van de literatuurstudie, requirements opgesteld. Dit omvat het vaststellen van +++++ (1-2weken)

2. Requirements, Long list en Short list (1-2 weken)
In deze fase wordt een duidelijk beeld gevormd van de vereisten voor het onderzoek. Dit omvat het identificeren van belangrijke aspecten van mobiele app-prestaties, zoals laadtijden, reactietijden en geheugenbeheer. De long list omvat alle mogelijke oplossingen en benaderingen, terwijl de short list de meest veelbelovende opties bevat.

Hieropvolgend wordt er een Proof-of-Concept (PoC) ontwikkeld en geïmplementeerd. Er zullen twee applicaties opgesteld worden, de een in React Native en de ander in Ionic. Voor Ionic zal er gekozen worden voor het framework React om en zo nauw mogelijk beeld te geven van de performantieverschillen tussen beide frameworks. Er zullen verschillende versies van de applicatie ontwikkeld worden met elk een andere focus. Zo kan er getest worden naar specifieke doeleinden zoals bijvoorbeeld werken met afbeeldingen, requests naar een API, etc. Voor het CPU-gebruik zullen profiler tools gebruikt worden om deze te meten. Dit kan aan de hand van React Devtools voor React Native en Chrome Devtools voor Ionic. Voor het geheugengebruik zal er gebruikt gemaakt worden van de Android Profiler in Android Studio. Tot slot zal er ook nog gekeken worden naar laadtijden en de opslag van het gecompileerde APK-bestand. Deze fase zal ongeveer 4-5 weken duren.

In de voorlaatste fase van de methodologie worden de resultaten van de PoC geanalyseerd en vergeleken. Hierbij wordt er gekeken naar de verschillen in performantie tussen beide frameworks. Deze fase zal ongeveer 4 weken duren.++++

Tot slot worden alle bevindingen gebundeld in een aaneensluitende scriptie. Deze zal een antwoord bieden op de centrale onderzoeksvraag. De uiteindelijke uiteenschrijving zal een vier- tot vijftal weken in beslag nemen.

Hier beschrijf je hoe je van plan bent het onderzoek te voeren. Welke onderzoekstechniek ga je toepassen om elk van je onderzoeksvragen te beantwoorden? Gebruik je hiervoor literatuurstudie, interviews met belanghebbenden (bv.~voor requirements-analyse), experimenten, simulaties, vergelijkende studie, risico-analyse, PoC, \ldots?

Valt je onderwerp onder één van de typische soorten bachelorproeven die besproken zijn in de lessen Research Methods (bv.\ vergelijkende studie of risico-analyse)? Zorg er dan ook voor dat we duidelijk de verschillende stappen terug vinden die we verwachten in dit soort onderzoek!

Vermijd onderzoekstechnieken die geen objectieve, meetbare resultaten kunnen opleveren. Enquêtes, bijvoorbeeld, zijn voor een bachelorproef informatica meestal \textbf{niet geschikt}. De antwoorden zijn eerder meningen dan feiten en in de praktijk blijkt het ook bijzonder moeilijk om voldoende respondenten te vinden. Studenten die een enquête willen voeren, hebben meestal ook geen goede definitie van de populatie, waardoor ook niet kan aangetoond worden dat eventuele resultaten representatief zijn.

Uit dit onderdeel moet duidelijk naar voor komen dat je bachelorproef ook technisch voldoen\-de diepgang zal bevatten. Het zou niet kloppen als een bachelorproef informatica ook door bv.\ een student marketing zou kunnen uitgevoerd worden.

Je beschrijft ook al welke tools (hardware, software, diensten, \ldots) je denkt hiervoor te gebruiken of te ontwikkelen.

Probeer ook een tijdschatting te maken. Hoe lang zal je met elke fase van je onderzoek bezig zijn en wat zijn de concrete \emph{deliverables} in elke fase?

%---------- Verwachte resultaten ----------------------------------------------
\section{Verwacht resultaat, conclusie}%
\label{sec:verwachte_resultaten}

Hier beschrijf je welke resultaten je verwacht. Als je metingen en simulaties uitvoert, kan je hier al mock-ups maken van de grafieken samen met de verwachte conclusies. Benoem zeker al je assen en de onderdelen van de grafiek die je gaat gebruiken. Dit zorgt ervoor dat je concreet weet welk soort data je moet verzamelen en hoe je die moet meten.

Wat heeft de doelgroep van je onderzoek aan het resultaat? Op welke manier zorgt jouw bachelorproef voor een meerwaarde?

Hier beschrijf je wat je verwacht uit je onderzoek, met de motivatie waarom. Het is \textbf{niet} erg indien uit je onderzoek andere resultaten en conclusies vloeien dan dat je hier beschrijft: het is dan juist interessant om te onderzoeken waarom jouw hypothesen niet overeenkomen met de resultaten.

