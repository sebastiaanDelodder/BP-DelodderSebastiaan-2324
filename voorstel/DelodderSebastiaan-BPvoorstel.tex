%==============================================================================
% Sjabloon onderzoeksvoorstel bachproef
%==============================================================================
% Gebaseerd op document class `hogent-article'
% zie <https://github.com/HoGentTIN/latex-hogent-article>

% Voor een voorstel in het Engels: voeg de documentclass-optie [english] toe.
% Let op: kan enkel na toestemming van de bachelorproefcoördinator!
\documentclass{hogent-article}

% Invoegen bibliografiebestand
\addbibresource{voorstel.bib}

% Informatie over de opleiding, het vak en soort opdracht
\studyprogramme{Professionele bachelor toegepaste informatica}
\course{Bachelorproef}
\assignmenttype{Onderzoeksvoorstel}
% Voor een voorstel in het Engels, haal de volgende 3 regels uit commentaar
% \studyprogramme{Bachelor of applied information technology}
% \course{Bachelor thesis}
% \assignmenttype{Research proposal}

\academicyear{2023-2024} % TODO: pas het academiejaar aan

% TODO: Werktitel
\title{Optimalisatie van mobiele app performantie bij streaming-applicaties: React Native vs Ionic op het Android-platform.}

% TODO: Studentnaam en emailadres invullen
\author{Sebastiaan Delodder}
\email{sebastiaan.delodder@student.hogent.be}
\projectrepo{https://github.com/sebastiaanDelodder/BP-DelodderSebastiaan-2324}

% TODO: Medestudent
% Gaat het om een bachelorproef in samenwerking met een student in een andere
% opleiding? Geef dan de naam en emailadres hier
% \author{Yasmine Alaoui (naam opleiding)}
% \email{yasmine.alaoui@student.hogent.be}

% TODO: Geef de co-promotor op
%\supervisor[Co-promotor]{S. Beekman (Synalco, \href{mailto:sigrid.beekman@synalco.be}{sigrid.beekman@synalco.be})}

% Binnen welke specialisatierichting uit 3TI situeert dit onderzoek zich?
% Kies uit deze lijst:
%
% - Mobile \& Enterprise development
% - AI \& Data Engineering
% - Functional \& Business Analysis
% - System \& Network Administrator
% - Mainframe Expert
% - Als het onderzoek niet past binnen een van deze domeinen specifieer je deze
%   zelf
%
\specialisation{Mobile \& Enterprise development}
\keywords{Android, Hybrid Mobile Framework, React Native, Ionic, Streaming}

\begin{document}

\begin{abstract}
  Streaming-applicaties zijn de laatste jaren steeds populairder geworden door enerzijds de opkomst van services zoals Netflix en Disney+, anderzijds door onze grote afhankelijkheid van mobiele apparaten. Het ontwikkelen van een kwaliteitsvolle streaming-applicatie brengt echter uitdagingen met zich mee. Allereerst moet de applicatie op een snel tempo videobeelden kunnen inladen en afspelen met hoge kwaliteit. Daarnaast kampen kleinere softwarebedrijven en start-ups vaak met een beperkt budget voor het ontwikkelen van hun eerste mobiele applicatie. In deze situatie wordt het vrij moeilijk om dezelfde applicatie te ontwikkelen voor meerdere platformen tegelijk, met elk hun eigen programmeertaal en ontwikkelingomgeving. Om dit probleem op te lossen, zijn er verschillende cross-platform frameworks beschikbaar die het mogelijk maken om eenmalig een applicatie te coderen om vervolgens te lanceren op verschillende besturingssystemen. Dit onderzoeksvoorstel richt zich bijgevolg op de performantie van mobiele app-prestaties, met specifieke aandacht voor de ontwikkeling in React Native en Ionic op Android toestellen. Er is een grote noodzaak om mobiele apps zo snel en efficiënt mogelijk te laten werken vanwege de lage processorcapaciteit van mobiele apparaten. Er zal een Proof-of-Concept worden uitgevoerd waaruit twee applicaties zullen voortvloeien, één in React Native en één in Ionic, die beiden over streaming-functionaliteiten zullen beschikken. Op basis van deze PoC zal een vergelijkende studie worden uitgevoerd om de prestaties van beide frameworks te vergelijken op vlak van CPU-gebruik, geheugengebruik en reactietijd. Hieruit zal blijken dat React Native een kleiner voordeel zal hebben in vergelijking met Ionic. De vraag is natuurlijk als dit (kleine) voordeel toch een beslissende factor kan spelen bij de keuze van een framework, of eerder verwaarloosbaar is bij het ontwikkelen van streaming-applicaties.
\end{abstract}

\tableofcontents

% De hoofdtekst van het voorstel zit in een apart bestand, zodat het makkelijk
% kan opgenomen worden in de bijlagen van de bachelorproef zelf.
%---------- Inleiding ---------------------------------------------------------

\section{Introductie}%
\label{sec:introductie}

Op vlak van mobiele app-ontwikkeling is er altijd een tweestrijd tussen het ontwikkelen voor Android en/of iOS. Het ontwikkelen voor beide platformen op een Native-aanpak voor dezelfde applicatie is echter een kostelijke en tijdrovende onderneming. De meeste Android applicaties worden geschreven met een combinatie van Java en Kotlin, terwijl iOS applicaties geschreven zijn in de door Apple ontwikkelde programmeertaal Swift. Het Native programmeren, d.w.z. het ontwikkelen van een applicatie specifiek bedoeld om te draaien op een bepaald platform, zorgt ervoor dat Swift-applicaties niet kunnen draaien op Android-platformen en omgekeerd. Vandaar het kostelijke aspect van het ontwikkelen voor beide platformen.

Cross-platform ontwikkelen biedt een oplossing voor dit probleem. De applicatie wordt hierbij één keer geschreven en kan vervolgens op verschillende platformen draaien. Dit wordt dan geprogrammeerd in talen die op verschillende platformen kunnen draaien zoals bijvoorbeeld JavaScript en C#. Hierbij is het grote nadeel dat de applicatie niet volledig is geoptimaliseerd voor het platform waarop het draait. Dit kan zelf leiden tot een tragere applicatie en een slechtere gebruikerservaring.

Dit bachelorproef zal zich focussen op het vergelijken van twee cross-platform frameworks, namelijk React Native en Ionic, op het Android-platform. React Native is een framework dat ontwikkeld is door Facebook en Ionic een framework dat ontwikkeld is door Drifty Co. Bij deze vergelijking zal er specifiek gekeken worden naar de performantie van de applicaties op basis van CPU-verbruik, laadtijden en geheugengebruik. De keuze voor deze specifieke frameworks is gebaseerd op de populariteit van deze frameworks, aangezien deze twee van de meest gebruikte cross-platform frameworks zijn.  Beide frameworks zijn bovendien open-source en gratis te gebruiken, wat de toegankelijkheid juist vergroot.

Hiervoor zal er een Proof-of-Concept worden uitgevoerd waarbij er een identieke applicatie zal worden ontwikkeld in React Native en Ionic. Deze applicaties zullen vervolgens getest worden op eerder vermelde performantie-aspecten. De resultaten van deze testen zullen tot slot worden geanalyseerd en vergeleken. Er zullen verschillende versies van deze applicaties worden ontwikkeld. Dit om uitvoerig en diepgaand te testen welke aspecten in welk framework een hogere impact hebben op de performantie. Zo kan er een uitgebreid beeld worden geschetst wat de voor- en nadelen zijn van beide frameworks.

Het onderzoek is gericht op ontwikkelaars en kleinere of middelgrote organisaties die met een beperkt budget hun applicatie op een zo groot mogelijke markt willen uitbrengen. Het onderzoek zal een antwoord bieden op de vraag welk framework het meest geschikt is voor het ontwikkelen van een performante applicatie op het Android-platform. Dit zal een meerwaarde bieden voor deze doelgroep, aangezien zij op basis van dit onderzoek een weloverwogen keuze kunnen maken voor het ontwikkelen van hun applicatie.


%---------- Stand van zaken ---------------------------------------------------

\section{State-of-the-art}%
\label{sec:state-of-the-art}

Hier beschrijf je de \emph{state-of-the-art} rondom je gekozen onderzoeksdomein, d.w.z.\ een inleidende, doorlopende tekst over het onderzoeksdomein van je bachelorproef. Je steunt daarbij heel sterk op de professionele \emph{vakliteratuur}, en niet zozeer op populariserende teksten voor een breed publiek. Wat is de huidige stand van zaken in dit domein, en wat zijn nog eventuele open vragen (die misschien de aanleiding waren tot je onderzoeksvraag!)?

Je mag de titel van deze sectie ook aanpassen (literatuurstudie, stand van zaken, enz.). Zijn er al gelijkaardige onderzoeken gevoerd? Wat concluderen ze? Wat is het verschil met jouw onderzoek?

Verwijs bij elke introductie van een term of bewering over het domein naar de vakliteratuur, bijvoorbeeld~\autocite{Hykes2013}! Denk zeker goed na welke werken je refereert en waarom.

Draag zorg voor correcte literatuurverwijzingen! Een bronvermelding hoort thuis \emph{binnen} de zin waar je je op die bron baseert, dus niet er buiten! Maak meteen een verwijzing als je gebruik maakt van een bron. Doe dit dus \emph{niet} aan het einde van een lange paragraaf. Baseer nooit teveel aansluitende tekst op eenzelfde bron.

Als je informatie over bronnen verzamelt in JabRef, zorg er dan voor dat alle nodige info aanwezig is om de bron terug te vinden (zoals uitvoerig besproken in de lessen Research Methods).

% Voor literatuurverwijzingen zijn er twee belangrijke commando's:
% \autocite{KEY} => (Auteur, jaartal) Gebruik dit als de naam van de auteur
%   geen onderdeel is van de zin.
% \textcite{KEY} => Auteur (jaartal)  Gebruik dit als de auteursnaam wel een
%   functie heeft in de zin (bv. ``Uit onderzoek door Doll & Hill (1954) bleek
%   ...'')

Je mag deze sectie nog verder onderverdelen in subsecties als dit de structuur van de tekst kan verduidelijken.

%---------- Methodologie ------------------------------------------------------
\section{Methodologie}%
\label{sec:methodologie}

Hier beschrijf je hoe je van plan bent het onderzoek te voeren. Welke onderzoekstechniek ga je toepassen om elk van je onderzoeksvragen te beantwoorden? Gebruik je hiervoor literatuurstudie, interviews met belanghebbenden (bv.~voor requirements-analyse), experimenten, simulaties, vergelijkende studie, risico-analyse, PoC, \ldots?

Valt je onderwerp onder één van de typische soorten bachelorproeven die besproken zijn in de lessen Research Methods (bv.\ vergelijkende studie of risico-analyse)? Zorg er dan ook voor dat we duidelijk de verschillende stappen terug vinden die we verwachten in dit soort onderzoek!

Vermijd onderzoekstechnieken die geen objectieve, meetbare resultaten kunnen opleveren. Enquêtes, bijvoorbeeld, zijn voor een bachelorproef informatica meestal \textbf{niet geschikt}. De antwoorden zijn eerder meningen dan feiten en in de praktijk blijkt het ook bijzonder moeilijk om voldoende respondenten te vinden. Studenten die een enquête willen voeren, hebben meestal ook geen goede definitie van de populatie, waardoor ook niet kan aangetoond worden dat eventuele resultaten representatief zijn.

Uit dit onderdeel moet duidelijk naar voor komen dat je bachelorproef ook technisch voldoen\-de diepgang zal bevatten. Het zou niet kloppen als een bachelorproef informatica ook door bv.\ een student marketing zou kunnen uitgevoerd worden.

Je beschrijft ook al welke tools (hardware, software, diensten, \ldots) je denkt hiervoor te gebruiken of te ontwikkelen.

Probeer ook een tijdschatting te maken. Hoe lang zal je met elke fase van je onderzoek bezig zijn en wat zijn de concrete \emph{deliverables} in elke fase?

%---------- Verwachte resultaten ----------------------------------------------
\section{Verwacht resultaat, conclusie}%
\label{sec:verwachte_resultaten}

Hier beschrijf je welke resultaten je verwacht. Als je metingen en simulaties uitvoert, kan je hier al mock-ups maken van de grafieken samen met de verwachte conclusies. Benoem zeker al je assen en de onderdelen van de grafiek die je gaat gebruiken. Dit zorgt ervoor dat je concreet weet welk soort data je moet verzamelen en hoe je die moet meten.

Wat heeft de doelgroep van je onderzoek aan het resultaat? Op welke manier zorgt jouw bachelorproef voor een meerwaarde?

Hier beschrijf je wat je verwacht uit je onderzoek, met de motivatie waarom. Het is \textbf{niet} erg indien uit je onderzoek andere resultaten en conclusies vloeien dan dat je hier beschrijft: het is dan juist interessant om te onderzoeken waarom jouw hypothesen niet overeenkomen met de resultaten.



\printbibliography[heading=bibintoc]

\end{document}