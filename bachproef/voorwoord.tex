%%=============================================================================
%% Voorwoord
%%=============================================================================

\chapter*{\IfLanguageName{dutch}{Woord vooraf}{Preface}}%
\label{ch:voorwoord}

%% TODO:
%% Het voorwoord is het enige deel van de bachelorproef waar je vanuit je
%% eigen standpunt (``ik-vorm'') mag schrijven. Je kan hier bv. motiveren
%% waarom jij het onderwerp wil bespreken.
%% Vergeet ook niet te bedanken wie je geholpen/gesteund/... heeft

Deze bachelorproef verdiept zich in de wereld van streaming-applicaties en meer bepaald de ontwikkeling ervan via Cross-platform en Hybrid frameworks. Tijdens mijn specialisatie in Mobile \& Enterprise Development, zagen wij vooral het native programmeren van applicaties. Hoewel dit zeker en vast zijn voordelen heeft, zorgt dit ervoor dat indien de applicatie voor meerdere platformen bestemd is, deze telkens meerdere keren geïmplementeerd moeten worden.

Dit bracht mij tot het onderwerp van deze bachelorproef. Al vrij snel bleek dan ook dat er nog veel misconcepties bestaan omtrent Hybrid frameworks en dat deze vaak nog steeds als inferieur worden beschouwd ten opzichte van Cross-Platform frameworks. Een groot deel van deze misconcepties zijn gebaseerd op onderzoeken die reeds enkele jaren oud zijn en ondertussen vrij achterhaald zijn. Ik vond het dan ook interessant om mijzelf te verdiepen in dit onderwerp en hier mijn steentje aan bij te dragen.

Ik hoop vooral dat mijn bachelorproef kan bijdragen aan de voordelen van beide frameworks en dat organisaties of ontwikkelaars die misschien niet beschikken over de juiste financiële middelen om voor elk platform een native applicatie te ontwikkelen, toch kunnen overtuigd worden om te kiezen voor een Hybrid of Cross-Platform framework.

Tot slot zou ik graag mijn promotor, mevrouw Vandermeersch, willen bedanken voor de uitstekende begeleiding en feedback die ik altijd heb mogen ontvangen. Het feit dat ik altijd bij haar terecht kon met vragen of problemen, heeft ervoor gezorgd dat ik mijn onderzoek tot een goed einde heb kunnen brengen.