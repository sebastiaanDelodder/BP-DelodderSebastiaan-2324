%%=============================================================================
%% Inleiding
%%=============================================================================

\chapter{\IfLanguageName{dutch}{Inleiding}{Introduction}}%
\label{ch:inleiding}

%De inleiding moet de lezer net genoeg informatie verschaffen om het onderwerp te begrijpen en in te zien waarom de onderzoeksvraag de moeite waard is om te onderzoeken. In de inleiding ga je literatuurverwijzingen beperken, zodat de tekst vlot leesbaar blijft. Je kan de inleiding verder onderverdelen in secties als dit de tekst verduidelijkt. Zaken die aan bod kunnen komen in de inleiding~\autocite{Pollefliet2011}:

%\begin{itemize}
%  \item context, achtergrond
%  \item afbakenen van het onderwerp
%  \item verantwoording van het onderwerp, methodologie
%  \item probleemstelling
%  \item onderzoeksdoelstelling
%  \item onderzoeksvraag
%  \item \ldots
%\end{itemize}

\section{\IfLanguageName{dutch}{Probleemstelling}{Problem Statement}}%
\label{sec:probleemstelling}

%Uit je probleemstelling moet duidelijk zijn dat je onderzoek een meerwaarde heeft voor een concrete doelgroep. De doelgroep moet goed gedefinieerd en afgelijnd zijn. Doelgroepen als ``bedrijven,'' ``KMO's'', systeembeheerders, enz.~zijn nog te vaag. Als je een lijstje kan maken van de personen/organisaties die een meerwaarde zullen vinden in deze bachelorproef (dit is eigenlijk je steekproefkader), dan is dat een indicatie dat de doelgroep goed gedefinieerd is. Dit kan een enkel bedrijf zijn of zelfs één persoon (je co-promotor/opdrachtgever).

Streaming-applicaties zijn de laatste jaren steeds populairder geworden door enerzijds de opkomst van services zoals Netflix en Disney+, anderzijds door onze grote afhankelijkheid van mobiele apparaten. Het ontwikkelen van een kwaliteitsvolle streaming-applicatie brengt echter uitdagingen met zich mee. Allereerst moet de applicatie op een snel tempo videobeelden kunnen inladen en afspelen met hoge kwaliteit. Daarnaast kampen kleinere softwarebedrijven en start-ups vaak met een beperkt budget voor het ontwikkelen van hun eerste mobiele applicatie.

In deze situatie wordt het vrij moeilijk om dezelfde applicatie te ontwikkelen voor meerdere platformen tegelijk, met elk hun eigen programmeertaal en ontwikkelingomgeving. Om dit probleem echter op te lossen, zijn er verschillende cross-platform frameworks beschikbaar die het mogelijk maken om eenmalig een applicatie te coderen om vervolgens te lanceren op verschillende besturingssystemen. Dit onderzoeksvoorstel richt zich bijgevolg op de performantie van mobiele app-prestaties, met specifieke aandacht voor de ontwikkeling in React Native en Ionic op Android toestellen. Er is een grote noodzaak om mobiele apps zo snel en efficiënt mogelijk te laten werken vanwege de lage processorcapaciteit van mobiele apparaten.

\section{\IfLanguageName{dutch}{Onderzoeksvraag}{Research question}}%
\label{sec:onderzoeksvraag}

%Wees zo concreet mogelijk bij het formuleren van je onderzoeksvraag. Een onderzoeksvraag is trouwens iets waar nog niemand op dit moment een antwoord heeft (voor zover je kan nagaan). Het opzoeken van bestaande informatie (bv. ``welke tools bestaan er voor deze toepassing?'') is dus geen onderzoeksvraag. Je kan de onderzoeksvraag verder specifiëren in deelvragen. Bv.~als je onderzoek gaat over performantiemetingen, dan 

De onderzoeksvraag luidt als volgt: Hoe verhouden de frameworks React Native en Ionic zich tot elkaar in termen van prestatieoptimalisatie voor streaming-applicaties op het Android-platform? Er wordt hierbij specifiek gekeken naar de snelheid, efficiëntie en algehele performantie van de mobiele applicatie bij het inladen, verwerken en afspelen van videobeelden. Het is belangrijk hierbij te vermelden dat de video-compilers en -decoders niet worden meegenomen in dit onderzoek, aangezien dit niet betrekking heeft op de frameworks zelf, maar eerder op het Android besturingssysteem en dus buiten de scope van dit onderzoek valt.

\section{\IfLanguageName{dutch}{Onderzoeksdoelstelling}{Research objective}}%
\label{sec:onderzoeksdoelstelling}

%Wat is het beoogde resultaat van je bachelorproef? Wat zijn de criteria voor succes? Beschrijf die zo concreet mogelijk. Gaat het bv.\ om een proof-of-concept, een prototype, een verslag met aanbevelingen, een vergelijkende studie, enz.

Uit het onderzoek zal een Proof-of-Concept voortvloeien waaruit twee identieke applicaties zullen worden ontwikkeld, één in React Native en één in Ionic. Beiden zullen getest worden op CPU-gebruik, geheugengebruik, reactietijd en snelheid van het inladen van videobeelden. Vanwege de verschillende architectuur van beide frameworks, zullen beiden dit op een andere manier aanpakken. Het doel is om kritieke punten te identificeren waarop de frameworks zich van elkaar onderscheiden, en afvragen en onderzoeken wat de achterliggende reden hiervan is.

Aan de hand van deze resultaten en bevindingen, kan het bedrijven en ontwikkelaars helpen om een doordachte keuze te maken bij het kiezen van een framework. Hoewel dit onderzoek zich specifiek richt op streaming-applicaties, kunnen de bevindingen ook doorgetrokken worden naar de algemene performantie tussen React Native en Ionic. Daarnaast zullen de conclusies ook inzicht bieden in de impact en voordelen van cross-platform en hybrid frameworks. Enerzijds vereenvoudigen deze frameworks de ontwikkeling van mobiele applicaties voor meerdere platformen, anderzijds kampen zij vaak met mindere prestaties in vergelijking met native applicaties. Toch kan het voor bepaalde projecten een interessante keuze zijn omwille van de snellere ontwikkeltijd en lagere kosten.

\section{\IfLanguageName{dutch}{Opzet van deze bachelorproef}{Structure of this bachelor thesis}}%
\label{sec:opzet-bachelorproef}

% Het is gebruikelijk aan het einde van de inleiding een overzicht te
% geven van de opbouw van de rest van de tekst. Deze sectie bevat al een aanzet
% die je kan aanvullen/aanpassen in functie van je eigen tekst.

(Dit onderdeel moet nog verder aangevuld worden)

De rest van deze bachelorproef is als volgt opgebouwd: 

In Hoofdstuk~\ref{ch:stand-van-zaken} wordt een overzicht gegeven van de stand van zaken binnen het onderzoeksdomein, op basis van een literatuurstudie.

In Hoofdstuk~\ref{ch:methodologie} wordt de methodologie toegelicht en worden de gebruikte onderzoekstechnieken besproken om een antwoord te kunnen formuleren op de onderzoeksvragen.

% TODO: Vul hier aan voor je eigen hoofstukken, één of twee zinnen per hoofdstuk

%In Hoofdstuk~\ref{ch:conclusie}, tenslotte, wordt de conclusie gegeven en een antwoord geformuleerd op de onderzoeksvragen. Daarbij wordt ook een aanzet gegeven voor toekomstig onderzoek binnen dit domein.