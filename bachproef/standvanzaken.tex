\chapter{\IfLanguageName{dutch}{Stand van zaken}{State of the art}}%
\label{ch:stand-van-zaken}

% Tip: Begin elk hoofdstuk met een paragraaf inleiding die beschrijft hoe
% dit hoofdstuk past binnen het geheel van de bachelorproef. Geef in het
% bijzonder aan wat de link is met het vorige en volgende hoofdstuk.

% Pas na deze inleidende paragraaf komt de eerste sectiehoofding.

In dit hoofdstuk wordt de huidige stand van zaken besproken+++. Er wordt eerst gesproken over het verschil tussen Native en Cross-Platform development. Het is namelijk noodzakelijk hier een goed beeld van te vormen om de juiste omkadering te geven van het onderzoek. Vervolgens wordt er dieper ingegaan op de Android infrastructuur. Vanwege de verschillen tussen de infrastructuur en werking van React Native en Ionic, is het belangrijk ook een begrip te hebben van de onderliggende infrastructuur. Vervolgens zal er op dit vlak een vergelijking gemaakt worden tussen deze twee frameworks en zullen er verklaringen of redenen gegeven worden op welke vlakken zij kunnen verschillen en waarom. Hoe deze aspecten in hun geheel een impact zullen hebben op de algemene performantie, zal in een later hoofdstuk (TODO: referentie naar hoofdstuk) besproken worden. Tot slot zullen de +++++++++

\paragraph{Native vs Cross-Platform vs Hybrid}

Voordat er dieper ingegaan wordt op React Native en Ionic, is het belangrijk om ze te situeren binnen Mobile Development. Er kunnen drie grote categorieën onderscheiden worden: Native, Cross-Platform en Hybrid. Native Development is het "native" ontwikkelen van een applicatie voor een specifiek platform. Dit wil zeggen dat een applicatie ontwikkeld wordt in de programmeertaal van het platform, zoals Java of Kotlin voor Android-toestellen en Objective-C of Swift voor iOS-toestellen (bron 11, bron 1). Met andere woorden maken deze applicaties gebruik van de officiële ontwikkelingsomgevingen en architecturen van de platform in kwestie. Dit brengt een aantal voordelen met zich mee. Door het gebruik van de native-architectuur en native-UI-tools, kan men profiteren van alle functionaliteiten die het platform aanbiedt. Dat zorgt dan weer voor een optimale performantie en bijgevolg een betere gebruikservaring. Bovendien vereisen vele type applicaties een native ontwikkeling. Denk hierbij aan applicaties die gebruik maken van de camera, GPS, gaming, enzoverder (bron). De keuze voor een native-aanpak brengt echter toch nadelen met zich mee. Mobiele applicaties worden echter vaak ontwikkeld met meerdere platformen in gedachten, waardoor dezelfde app als het ware dubbel moet worden geprogrammeerd, met als gevolg hogere kosten en development tijd??(bron). Bovendien vereist dit extra kennis van de medewerkers en ontwikkelaars omdat er voor elk platform een andere programmeertaal gebruikt wordt. Kleinere organisaties of start-ups hebben niet altijd voldoende middelen om zulke projecten te financieren. Vandaar dat Cross-Platform en Hybrid Development een oplossing kunnen bieden.

Het grote voordeel aan Cross-Platform development is dat het gebruik maakt van het principe: "write once, run everywhere" (bron??). Dit wil zeggen dat ontwikkelaars de applicatie slechts één keer moeten schrijven in een programmeertaal naar keuze en vervolgens uitrollen op één of meerdere platformen zonder hierbij platform-specifieke code te moeten herschrijven (bron). In vele gevallen maakt men gebruik van een framework dat gebaseerd is op talen die ondersteund worden op meerdere platformen, zoals bijvoorbeeld JavaScript. Vervolgens maakt het framework gebruik van een cross-compiler die de code vertaalt naar de platformspecifieke code. De applicatie heeft bijgevolg een soort van "native" look-and-feel, onder meer omdat Native-componenten van het toestel hergebruikt worden. Belangrijk hierbij te vermelden is dat het nog altijd geen echte Native-applicatie is vanwege de cross-compiler. Deze tussenstap brengt namelijk een zekere overhead met zich mee, wat kan leiden tot een mindere performantie in vergelijking met een Native-applicatie. Bovendien is het niet altijd mogelijk om alle platformspecifieke functionaliteiten te gebruiken. Wat deze beperkingen zijn, hangt natuurlijk af van framework tot framework. Wat wel vaststaat is dat Cross-Platform Development, in tegenstelling tot Native Development, een pak goedkoper is en sneller verloopt juist vanwege het feit dat er slechts één codebase is.

Tot slot is er nog Hybrid Development. Deze vorm bevindt zich ergens tussen Native en Cross-Platform Development. Uit verschillende bronnen bleek dat de termen Cross-Platform en Hybrid Development vaak door elkaar worden gebruikt, terwijl er toch een aanzienlijk verschil is tussen beiden. Hybrid Development maakt gebruik van webtechnologieën zoals HTML, CSS en JavaScript om een applicatie te ontwikkelen. Er is hier dus opnieuw sprake van herbruikbare code en dus het eenmalig implementeren van de applicatie. Het grote verschil met Cross-Platform Development is dat Hybrid Development de code in een webview of browser-engine van het platform plaatst. Bij Android is dit het WebView component, bij iOS wordt het UIWebView component gebruikt (bron4). Vervolgens wordt de applicatie verpakt in een native container waardoor de applicatie een soort van "native" applicatie wordt. Toch valt hierbij op te merken dat de applicatie niet Native is, en ook geen gebruik maakt van Native-componenten zoals bij Cross-Platform Development. Omdat Hybrid applicaties dus ook geen gebruik maken van Native-componenten, kampen zij vaak met inconsistente User Interface, wat natuurlijk een impact heeft op de gebruikerservaring.

Het verschil tussen Cross-Platform en Hybrid development vormt een kernaspect binnen dit onderzoek. Ionic wordt namelijk bestempeld als een Hybrid framework, terwijl React Native een Cross-Platform framework is. Beiden zijn echter mogelijkheden om eenmalig een programma te coderen en vervolgens uit te rollen op meerdere platformen. Beiden zijn echter ook niet Native, met de bijhorende impact op de performantie. Hoe de infrastructuur van beide frameworks in elkaar zit, zal in een volgende paragraaf besproken worden.


\paragraph{Android infrastructuur}

\paragraph{React Native}

\paragraph{Ionic}

\paragraph{Performantie}

Dit hoofdstuk bevat je literatuurstudie. De inhoud gaat verder op de inleiding, maar zal het onderwerp van de bachelorproef *diepgaand* uitspitten. De bedoeling is dat de lezer na lezing van dit hoofdstuk helemaal op de hoogte is van de huidige stand van zaken (state-of-the-art) in het onderzoeksdomein. Iemand die niet vertrouwd is met het onderwerp, weet nu voldoende om de rest van het verhaal te kunnen volgen, zonder dat die er nog andere informatie moet over opzoeken \autocite{Pollefliet2011}.

Je verwijst bij elke bewering die je doet, vakterm die je introduceert, enz.\ naar je bronnen. In \LaTeX{} kan dat met het commando \texttt{$\backslash${textcite\{\}}} of \texttt{$\backslash${autocite\{\}}}. Als argument van het commando geef je de ``sleutel'' van een ``record'' in een bibliografische databank in het Bib\LaTeX{}-formaat (een tekstbestand). Als je expliciet naar de auteur verwijst in de zin (narratieve referentie), gebruik je \texttt{$\backslash${}textcite\{\}}. Soms is de auteursnaam niet expliciet een onderdeel van de zin, dan gebruik je \texttt{$\backslash${}autocite\{\}} (referentie tussen haakjes). Dit gebruik je bv.~bij een citaat, of om in het bijschrift van een overgenomen afbeelding, broncode, tabel, enz. te verwijzen naar de bron. In de volgende paragraaf een voorbeeld van elk.

\textcite{Knuth1998} schreef een van de standaardwerken over sorteer- en zoekalgoritmen. Experten zijn het erover eens dat cloud computing een interessante opportuniteit vormen, zowel voor gebruikers als voor dienstverleners op vlak van informatietechnologie~\autocite{Creeger2009}.

Let er ook op: het \texttt{cite}-commando voor de punt, dus binnen de zin. Je verwijst meteen naar een bron in de eerste zin die erop gebaseerd is, dus niet pas op het einde van een paragraaf.

\lipsum[7-20]
